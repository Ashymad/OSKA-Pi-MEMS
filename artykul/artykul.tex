\documentclass[a4paper,12pt]{mwart}

\usepackage[polish,english]{babel}
\usepackage[utf8]{inputenc}
\usepackage{polski}
\usepackage[T1]{fontenc}
\usepackage{indentfirst}
\frenchspacing

\usepackage{enumerate}
\usepackage{graphicx}
\usepackage{float}
\usepackage{makecell}
\usepackage{siunitx}
\sisetup{output-decimal-marker = {,}}
\usepackage{icomma}
\let\lll\undefined
\usepackage{amsmath, amssymb, amsfonts}
\usepackage{mathtools}
\usepackage{setspace}

\begin{document}

\onehalfspacing
\begin{flushleft}

  \textbf{Szymon \MakeUppercase{Mikulicz}} \\
  \textbf{Marcel \MakeUppercase{Piszak}} \\
  \vspace*{12pt}

  \MakeUppercase{\textbf{Wibroakustyczna stacja pomiarowa oparta o mikrokomputer Raspberry Pi}} \\
  \vspace*{6pt}
  \MakeUppercase{\textbf{Vibroacoustic measuring station based on Raspberry Pi microcomputer}} \\
  \vspace*{12pt}

  AGH Akademia Górniczo-Hutnicza \\
  Wydział Inżynierii Mechanicznej i Robotyki \\
  Katedra Mechaniki i Wibroakustyki \\
  \vspace*{6pt}

  czilukim@o2.pl \\
  marcel.piszak@wp.pl \\

\end{flushleft}

\noindent
\textbf{Streszczenie} \\
Treść streszczenia
\vspace*{24pt}

\noindent
\textbf{Abstrakt} \\
Współczesne rozwiązania dedykowane do pomiarów drgań budynków
proponowane przez popularnych producentów sprzętu pomiarowego
charakteryzują się dużym stopniem skomplikowania. Celem niniejszego
projektu było zbudowanie prostego i przystępnego cenowo systemu
pozwalającego na pomiar i analizę drgań budynków zgodną z normą PN-B
02170:2016-12. Postanowiono stworzyć system pomiarowy oparty na
platformie Raspberry Pi® oraz wykorzystujący akcelerometry typu MEMS w
celu minimalizacji kosztów i rozmiarów stacji. Oprogramowanie
kontrolujące akwizycję oraz analizę danych napisano przy użyciu języka
Python i odpowiednich bibliotek. Skonstruowanie oraz uruchomienie
zaprogramowanego układu ma bezpośrednio przyczynić się do realizacji
większego projektu obejmującego połączenie takich jednostek analizujący
drgania na większym obszarze.
\vspace*{24pt}

\section{Wstęp}

\section{}

\section{Wnioski}

\end{document}
